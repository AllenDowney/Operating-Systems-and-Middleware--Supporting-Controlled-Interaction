%% This file is a portion of the source for Revised Edition 1.1 of
%% Operating Systems and Middleware: Supporting Controlled
%% Interaction, Copyright 2011 by Max Hailperin.  This work is
%% licensed under the Creative Commons Attribution-ShareAlike 3.0
%% Unported License. To view a copy of this license, visit
%% http://creativecommons.org/licenses/by-sa/3.0/ or send a letter to
%% Creative Commons, 171 Second Street, Suite 300, San Francisco,
%% California, 94105, USA.
\chapter*{Preface}\markboth{PREFACE}{PREFACE}

Suppose you
sit down at your computer to check your email. One of the messages
includes an attached document, which you are to edit.  You click
the attachment, and it opens up in another window.  After you start
editing the document, you realize you need to leave for a trip.  You
save the document in its partially edited state and shut down the
computer to save energy while you are gone.  Upon returning, you boot
the computer back up, open the document, and continue editing.

This scenario illustrates that computations interact.  In fact, it demonstrates at least three kinds of interactions
between computations.  In each case, one computation provides data to
another.  First, your email program retrieves new mail from the
server, using the Internet to bridge space.  Second, your email
program provides the attachment to the word processor, using the
operating system's services to couple the two application programs.
Third, the invocation of the
word processor that is running before your trip provides the partially edited
document to the invocation running after your return, using disk
storage to bridge time.

In this book, you will learn about all three kinds of interaction.  In
all three cases, interesting software techniques are needed in order
to bring the computations into contact, yet keep them sufficiently at
arm's length that they don't compromise each other's reliability.
The exciting challenge, then, is supporting controlled interaction.
This includes support for computations that share a single computer
and interact with one another, as your email and word processing
programs do.  It also includes support for data storage and network
communication.  This book describes how all these kinds of support are
provided both by operating systems and by additional software layered
on top of operating systems, which is known as middleware.

\section*{Audience}

If you are an upper-level computer science student who wants to
understand how contemporary operating systems and middleware products
work and why they work that way, this book is for you.  In this book,
you will find many forms of balance.  The high-level application
programmer's view, focused on the services that system software
provides, is balanced with a lower-level perspective, focused on the
mechanisms used to provide those services.  Timeless concepts are
balanced with concrete examples of how those concepts are embodied in
a range of currently popular systems.  Programming is balanced with
other intellectual activities, such as the scientific measurement of system
performance and the strategic consideration of system security in its
human and business context. Even the programming languages used for
examples are balanced, with some examples in Java and others in C or
C$++$.  (Only limited portions of these languages are used, however,
so that the examples can serve as learning opportunities, not
stumbling blocks.)

\section*{Systems Used as Examples}

Most of the examples throughout the book are drawn from the two
dominant families of operating systems: Microsoft Windows and the UNIX
family, including especially Linux and Mac OS~X.  Using this range of
systems promotes the students' flexibility.  It also allows a more
comprehensive array of concepts to be concretely illustrated, as the
systems embody fundamentally different approaches to some problems,
such as the scheduling of processors' time and the tracking of files'
disk space.

Most of the examples are drawn from the stable core portions of the
operating systems and, as such, are equally applicable to a range of
specific versions.  Whenever Microsoft Windows is mentioned without
further specification, the material should apply to Windows NT, Windows
2000, Windows XP, Windows Server 2003, Windows Vista, Windows 2008, and Windows 7.  All Linux
examples are from version 2.6, though much of the material applies to
other versions as well.  Wherever actual Linux source code is shown
(or whenever fine details matter for other reasons), the specific subversion of
2.6 is mentioned in the end-of-chapter notes.  Most of the Mac OS~X examples
originated with version 10.4, also known as Tiger, but should be applicable to other versions.

Where the book discusses the protection of each process's memory, one
additional operating system is brought into the mix of examples, in
order to illustrate a more comprehensive range of alternative
designs.  The IBM iSeries, formerly known as the AS/400, embodies an
interesting approach to protection that might see wider application
within current students' lifetimes.  Rather than giving each process
its own address space (as Linux, Windows, and Mac OS~X do), the
iSeries allows all processes to share a single address space and to
hold varying access permissions to individual objects within that
space.

Several middleware systems are used for examples as well.  The Oracle
database system is used to illustrate deadlock detection and recovery
as well as the use of atomic transactions.  Messaging systems appear
both as another application of atomic transactions and as an important
form of communication middleware, supporting distributed
applications.  The specific messaging examples are drawn from the IBM
WebSphere MQ system (formerly MQSeries) and the Java
Message Service (JMS) interface, which is part of Java~2 Enterprise
Edition (J2EE).  The other communication middleware examples are Java
RMI (Remote Method Invocation) and web services.  Web services are
explained in platform-neutral terms using the SOAP and WSDL standards,
as well as through a J2EE interface, JAX-RPC (Java API for XML-Based RPC).

\section*{Organization of the Text}

Chapter~\ref{intro-chapter} provides an overview of the text as a
whole, explaining what an operating system is, what middleware is, and
what sorts of support these systems provide for controlled
interaction.

The next nine chapters work through the varieties of controlled
interaction that are exemplified by the scenario at the beginning of
the preface: interaction between concurrent computations on the same
system (as between your email program and your word processor),
interaction across time (as between your word processor before your
trip and your word processor after your trip), and interaction across
space (as between your email program and your service provider's email
server).

The first of these three topics is controlled interaction between
computations operating at one time on a particular computer.  Before
such interaction can make sense, you need to understand how it is that
a single computer can be running more than one program, such as an
email program in one window and a word processing program in another.
Therefore, Chapter~\ref{threads-chapter} explains the fundamental
mechanism for dividing a computer's attention between concurrent
computations, known as threads.  Chapter~\ref{scheduling-chapter}
continues with the related topic of scheduling.  That is, if the
computer is dividing its time between computations, it needs to decide
which ones to work on at any moment.

With concurrent computations explained,
Chapter~\ref{synchronization-chapter} introduces controlled
interactions between them by explaining synchronization, which is
control over the threads' relative timing.  For example, this chapter
explains how, when your email program sends a document to your word
processor, the word processor can be constrained to read the document
only after the email program writes it.   One particularly important form of
synchronization, atomic transactions, is the topic of
Chapter~\ref{transactions-chapter}.  Atomic transactions are groups of
operations that take place as an indivisible unit; they are most
commonly supported by middleware, though they are also playing an
increasing role in operating systems.

Other than synchronization, the main way that operating systems
control the interaction between computations is by controlling their
access to memory.  Chapter~\ref{vm-chapter} explains how this is
achieved using the technique known as virtual memory.  That chapter
also explains the many other objectives this same technique can serve.
Virtual memory serves as the foundation for
Chapter~\ref{processes-chapter}'s topic, which is processes.  A process is the
fundamental unit of computation for protected access, just as a thread
is the fundamental unit of computation for concurrency.  A process is
a group of threads that share a protection environment; in particular,
they share the same access to virtual memory.

The next three chapters move outside the limitations of a single computer
operating in a single session.  First, consider the document stored
before a trip and available again after it.
Chapter~\ref{persistence-chapter} explains persistent storage
mechanisms, focusing particularly on the file storage that operating
systems provide.  Second, consider the interaction between your
email program and your service provider's email server.
Chapter~\ref{networking-chapter} provides an overview of networking,
including the services that operating systems make available to
programs such as the email client and server.
Chapter~\ref{distmid-chapter} extends this discussion into the more
sophisticated forms of support provided by communication middleware,
such as messaging systems, RMI, and web services.

Finally, Chapter~\ref{security-chapter} focuses on security.  Because
security is a pervasive issue, the preceding ten chapters all provide
some information on it as well.  Specifically, the final section of
each chapter points out ways in which security relates to that
chapter's particular topic.  However, even with that coverage
distributed throughout the book, a chapter specifically on security is
needed, primarily to elevate it out of technical particulars and talk
about general principles and the human and organizational context
surrounding the computer technology.

The best way to use these chapters is in consecutive order.  However,
Chapter~\ref{transactions-chapter} can be omitted with only minor harm
to Chapters \ref{persistence-chapter} and \ref{distmid-chapter}, and
Chapter~\ref{networking-chapter} can be omitted if students are
already sufficiently familiar with networking.

\section*{Relationship to Computer Science Curriculum 2008}

Operating systems are traditionally the subject of a course required
for all computer science majors.  In recent years, however, there has
been increasing interest in the idea that upper-level courses should
be centered less around particular artifacts, such as operating
systems, and more around cross-cutting concepts.  In particular, the
{\em Computing Curricula 2001} (CC2001) and its interim revision,
\textit{Computer Science Curriculum 2008} (CS2008),
provide
encouragement for this approach, at least as one option.  Most
colleges and universities still retain a relatively traditional
operating systems course, however.  Therefore, this book steers a
middle course, moving in the direction of the cross-cutting concerns
while retaining enough familiarity to be broadly adoptable.

The following table indicates the placement within this text of
knowledge units from CS2008's computer science body of knowledge.
Those knowledge units designated as core units within CS2008 are listed in italics.
The book covers all core operating systems (OS) units, as well as one elective
OS unit.  The overall
amount of coverage for each unit is always at least that recommended by
CS2008, though sometimes the specific subtopics don't quite correspond exactly.
Outside the OS area, this book's most substantial coverage is of Net-Centric Computing (NC);
another major topic, transaction processing, comes from Information Management (IM).
In each row, the listed chapters
contain the bulk of the knowledge unit's coverage, though some topics may be
elsewhere.
\[\begin{tabular}{|l|l|}
\hline
\bf Knowledge unit &\\
\bf (italic indicates core units in CS2008)& \bf Chapter(s)\\\hline
\it OS/OverviewOfOperatingSystems & \ref{intro-chapter}\\
\it OS/OperatingSystemPrinciples &  \ref{intro-chapter}, \ref{processes-chapter}\\
\it OS/Concurrency & \ref{threads-chapter}, \ref{synchronization-chapter}\\
\it OS/SchedulingAndDispatch & \ref{scheduling-chapter}\\
\it OS/MemoryManagement & \ref{vm-chapter}\\
\it OS/SecurityAndProtection & \ref{processes-chapter}, \ref{security-chapter}\\
OS/FileSystems & \ref{persistence-chapter}\\
\it NC/Introduction & \ref{networking-chapter}\\
\it NC/NetworkCommunication (partial coverage)& \ref{networking-chapter}\\
\it NC/NetworkSecurity  (partial coverage)& \ref{networking-chapter}\\
NC/WebOrganization (partial coverage) & \ref{networking-chapter}\\
NC/NetworkedApplications (partial coverage) & \ref{distmid-chapter}\\
IM/TransactionProcessing & \ref{transactions-chapter}\\\hline
\end{tabular}\]

\section*{Your Feedback is Welcome}

Comments, suggestions, and bug reports are welcome; please send email to
\verb|max@gustavus.edu|.  Bug reports in particular can earn you a
bounty of \$2.56 apiece as a token of gratitude. (The great computer
scientist Donald Knuth started this tradition.  Given how close to
bug-free his publications have become, it seems to work.)  For
purposes of this reward, the definition of a bug is simple: if as a
result of your email the author chooses to make a change, then you
have pointed out a bug.  The change need not be the one you suggested,
and the bug need not be technical in nature.  Unclear writing
qualifies, for example.

\section*{Features of the Text}

Each chapter concludes with five standard elements.  The last numbered
section within the chapter is always devoted to security matters
related to the chapter's topic.  Next comes three different lists of
opportunities for active participation by the student: exercises,
programming projects, and exploration projects.  Finally, the chapter
ends with historical and bibliographic notes.

The distinction between exercises, programming projects, and
exploration projects needs explanation.  An exercise can be completed
with no outside resources beyond paper and pencil: you need just this
textbook and your mind.  That does not mean all the exercises are cut
and dried, however.  Some may call upon you to think creatively; for
these, no one answer is correct.  Programming projects require a
nontrivial amount of programming; that is, they require more than
making a small, easily identified change in an existing
program. However, a programming project may involve other activities
beyond programming.  Several of them involve scientific measurement of
performance effects, for example; these exploratory aspects may even
dominate over the programming aspects.  An exploration project, on the
other hand, can be an experiment that can be performed with no real
programming; at most you might change a designated line within an
existing program.  The category of exploration projects does not just
include experimental work, however.  It also includes projects that
require you to do research on the Internet or using other library
resources.

\section*{Supplemental Resources}

The author of this text is making supplemental resources available on
his own web site.  Additionally, the publisher of the
earlier first edition commissioned
additional resources from independent supplement authors,
which may still be available through the publisher's web site
and would largely still apply to this revised edition.
The author's web site, \textit{\url{http://gustavus.edu/+max/os-book/}},
contains at least the following materials:
\begin{itemize}
\item
Full text of this revised edition
\item
Source code in Java, C, or C$++$ for all programs
that are shown in the text
\item
Artwork files for all figures in the text
\item
An errata list that will be updated on an ongoing basis
\end{itemize}

\section*{About the Revised Edition}
Course Technology published the first edition of this book in January of 2006
and in October of 2010 assigned the copyright back to the author, giving him
the opportunity to make it freely available.  This revised edition closely follows
the first edition; rather than being a thorough update, it is aimed at three narrow goals:
\begin{itemize}
\item All errata reported in the first edition are corrected.
\item A variety of other minor improvements appear throughout, such as clarified explanations and additional exercises, projects, and end-of-chapter notes.
\item Two focused areas received more substantial updates:
\begin{itemize}
\item The explanation of Linux's scheduler was completely replaced to correspond to the newer ``Completely Fair Scheduler'' (CFS), including its group scheduling feature.
\item A new section, \ref{nonblocking-synchronization-section}, was added on nonblocking synchronization.
\end{itemize}
\end{itemize}

In focusing on these limited goals, a key objective was to maintain as much compatibility with the first edition as possible.  Although page numbering changed, most other numbers stayed the same.  All new exercises and projects were added to the end of the corresponding lists for that reason.  The only newly added section, \ref{nonblocking-synchronization-section}, is near the end of its chapter; thus, the only changed section number is that the old Section~4.9 (``Security and Synchronization'') became \ref{synchronization-and-security-section}.  Only in Chapter \ref{synchronization-chapter} did any figure numbers change.

It is my hope that others will join me in making further updates and improvements to the text.  I am releasing it under a Creative Commons license that allows not just free copying, but also the freedom to make modifications, so long as the modified version is released under the same terms.  In order to such modifications practical, I'm not just releasing the book in PDF form, but also as a collection of LaTeX source files that can be edited and then run through the \texttt{pdflatex} program (along with \texttt{bibtex} and \texttt{makeindex}).  The source file collection also includes PDF files of all artwork figures; Course Technology has released the rights to the artwork they contracted to have redrawn.

If you produce a modified version of this text, the Creative Commons license allows you considerable flexibility in how you make your modified version available.  I would urge you to send it back to me (\verb|max@gustavus.edu|) so that I can add your version to the main web site--we will all benefit from having a central repository of progress.  Separate materials to supplement the text would also be welcome.  One category that occurs to me is animations or screencasts; the static figures in the text are rather limited. Another worthwhile project would be to transform the text into a more contribution-friendly form, such as a wiki.

\section*{Acknowledgments}

This book was made possible by financial and logistical support from
my employer, Gustavus Adolphus College, and moral support from my
family.  I would like to acknowledge the contributions of the
publishing team, especially developmental editor Jill Batistick and
Product Manager Alyssa Pratt.  I am also grateful to my students for
doing their own fair share of teaching.  I particularly appreciate the
often extensive comments I received from the following individuals,
each of whom reviewed one or more chapters: Dan Cosley, University of
Minnesota, Twin Cities; Allen Downey, Franklin W. Olin College of
Engineering; Michael Goldweber, Xavier University; Ramesh Karne,
Towson University; G. Manimaran, Iowa State University; Alexander
Manov, Illinois Institute of Technology; Peter Reiher, University of
California, Los Angeles; Rich Salz, DataPower Technology; Dave Schulz,
Wisconsin Lutheran College; Sanjeev Setia, George Mason University;
and Jon Weissman, University of Minnesota, Twin Cities.  Although I
did not adopt all their suggestions, I did not ignore any of them, and
I appreciate them all.

In preparing the revised edition, I took advantage of suggestions from many readers.  I would like to thank all of them, even those I've managed to lose track of, to whom I also apologize.  Those I can thank by name are Joel Adams, Michael Brackney, Jack Briner, Justin Delegard, Ben Follis, MinChan Kim, Finn Kuusisto, Matt Lindner, Milo Martin, Gabe Schmidt, Fritz Sieker, and Alex Wauck.
